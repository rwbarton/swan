\documentclass[11pt]{article}
 
\usepackage[margin=1in]{geometry} 
\usepackage{tikz-cd}
\usepackage{graphicx}
\usepackage{hyperref}
\usepackage{amsmath,amsthm,amssymb,mathrsfs,mathabx}
\usepackage{enumitem}
\usepackage{csquotes}
\usepackage{color} 
\usepackage{parskip}
\usepackage{hyperref}
\usepackage{fixltx2e}
\usepackage{tcolorbox}
\usepackage{listings}

\newcommand{\R}{\mathbb{R}}
\newcommand{\remph}[1]{\textcolor{red}{#1}}
\newcommand{\TODO}{\remph{TODO}}
\newcommand{\Hom}{\operatorname{Hom}}
\newcommand{\Mat}{\operatorname{Mat}}
\newcommand{\Vect}{\operatorname{Vect}}
\newcommand{\fgProj}{\operatorname{fgProj}}
\newcommand{\Shv}{\operatorname{Shv}}
\newcommand{\RHom}[3]{\Hom_{{#1}\text{-Mod}}({#2}, {#3})}
\newcommand{\im}{\operatorname{im}}
\renewcommand{\ker}{\operatorname{ker}}
\mathchardef\mhyphen="2D
\newcommand{\Mod}{\mhyphen\operatorname{Mod}}
\newcommand{\wt}{\widetilde}

\theoremstyle{plain}
\newtheorem{lemma}{Lemma}[section]
\newtheorem{theorem}{Theorem}[section]

\theoremstyle{definition}
\newtheorem{definition}{Definition}[section]
\newtheorem{example}{Example}[section]

\bibliographystyle{plainurl}

\usepackage{color}
\definecolor{keywordcolor}{rgb}{0.7, 0.1, 0.1}   % red
\definecolor{tacticcolor}{rgb}{0.1, 0.2, 0.6}    % blue
\definecolor{commentcolor}{rgb}{0.4, 0.4, 0.4}   % grey
\definecolor{symbolcolor}{rgb}{0.0, 0.1, 0.6}    % blue
\definecolor{sortcolor}{rgb}{0.1, 0.5, 0.1}      % green
\definecolor{attributecolor}{rgb}{0.7, 0.1, 0.1} % red

\usepackage{listings}
\def\lstlanguagefiles{lstlean.tex}
\lstset{language=lean,breakatwhitespace,xleftmargin=0pt, basicstyle=\ttfamily\small}
\usepackage{stmaryrd}
\newcommand{\B}{\mathbb{B}}
\newcommand{\lil}{\lstinline}
\newcommand{\N}{\mathbb{N}}

\begin{document}

\title{Swan's theorem}
\author{Jesse Han}
\date{\today}

\maketitle

\begin{abstract}
We give an account of Swan's theorem characterizing vector bundles over a compact Hausdorff space.
\end{abstract}

\section*{Introduction}
% TODO

\section{Vector bundles}
For the rest of this document, we fix a compact Hausdorff space \(X\).
% TODO: rephrase in terms of fiber functions
\begin{definition}

\label{def-vector-bundle}
 A (real) \textbf{vector bundle over \(X\)} comprises the following data:
  \begin{enumerate}[label={(\roman*)}]
  \item A topological space \(E\), called the \textbf{total space},
  \item a map of topological spaces \(E \overset{p}{\to} X\),
  \item on every \emph{fiber} \(E_x := p^{-1}(x)\), the structure of an \(\R\)-vector space on \(E_x\), and
  \item for every \(x : X\), an open neighborhood \(x \in U_x \subseteq X\), an \(n_x : \N\), and a homeomorphism \(\Phi_x : U_x \times \R^{n_x} \to p^{-1}(U)\),
  \end{enumerate}

  satisfying the following properties:
\begin{enumerate}[label={(\alph*)}]
  \item For every \(x : X\), the dimension of the vector space \(E_x\) is finite,
  \item \label{local-triviality} for every \(x : X\), \(\Phi_x\) is linear on every fiber, and satisfies \(p \circ \Phi_x = (\pi_1 : U_x \times \R^{n_x} \twoheadrightarrow U_x)\).
\end{enumerate}  
\end{definition}

\begin{example}\label{example-trivial-bundle}
  The property \ref{local-triviality} is called \textbf{local triviality}. Accordingly, for every \(n : \N\), we define the \textbf{trivial bundle} of rank \(n\) to be
  \[
    X \times \R^n \overset{\pi_1}{\twoheadrightarrow} X.
  \]
\end{example}

\begin{example}\label{example-tangent-bundle}
  Suppose that \(X\) additionally has the structure of a smooth manifold. The \textbf{tangent bundle} of \(X\) is defined as follows: the total space is sigma-type of all tangent spaces at all points of \(X\).

  \TODO(jesse): describe the topology, smooth structure, etc.
\end{example}

\begin{definition}\label{def-rank}
  Let \(E \overset{p}{\to} X\) be a vector bundle. Let \(x : X\). The \textbf{rank of \(E\) at \(x\)} is the dimension of the vector space \(E_x\).
\end{definition}

\begin{lemma}\label{lemma-rank-locally-constant}
  Rank is a locally constant function. That is, for every \(x : X\), there is a neighborhood \(V_{x}\) of \(x\) such that the rank of \(E\) is constant on \(V_x\).
\end{lemma}

\begin{proof}
  \TODO(jesse).
\end{proof}

\begin{definition} \label{def-vector-bundle-homomorphism}
  Let \(E_1 \overset{p_1}{\to} X\) and \(E_2 \overset{p_2}{\to} X\) be vector bundles over \(X\).
  A \textbf{vector bundle morphism} is a map \(f : E_1 \to E_2\) over \(X\) such that for every \(x : X\), \(E_{1,x} \overset{f}{\to} E_{2,f(x)}\) is also a map of vector spaces.
\end{definition}

\begin{definition} \label{def-category-vector-bundles}
  We define \(\Vect(X)\), the category of vector bundles over \(X\), to be the category whose objects are the vector bundles over \(X\) and whose morphisms are the vector bundle morphisms.
\end{definition}

\begin{definition}\label{def-global-sections}
  Let \(E \overset{p}{\to} X\) be a vector bundle over \(X\). The \textbf{global sections} of \(E\) are the set of continuous right-inverses to \(p\).
\end{definition}

\begin{definition}\label{def-global-sections-functor}
  More generally, we define the \textbf{global sections functor}
  \[\Gamma : \Vect(X) \to \R\Mod\]
  as follows:
  \TODO(jesse)
\end{definition}

\begin{lemma}[\cite{swan1962vector}, Lemma 1]\label{lemma-1}
  Let \(x \in X\). Let \(E \overset{p}{\to} X\) be a vector bundle, and let \(t_1, \dots, t_k\) be sections of \(E\) over a neighborhood \(U \ni x\) such that \(t_1(x), \dots, t_k(x)\) are linearly independent.

  Then there is a neighborhood \(V \ni x\) such that for all \(y \in V\), \(t_1(y), \dots, t_k(y)\) are linearly independent.
\end{lemma}

\begin{proof}
  Without loss of generality, we may assume that \(U\) trivializes \(E\) at \(x\), and thus that we are in the situation of having continuous functions
  \[s_1, \dots, s_k : U \to \R^n\]
  such that \(s_1(x), \dots, s_k(x)\) are linearly independent.
  
  We identify \((s_1(x), \dots, s_k(x))\) with a linear transformation \(f_x : \R^k \to \R^n\). It suffices to show that there is a neighborhood \(V\) of \(x\) such that \(\mathbf{s} := (x \mapsto f_x)\) (which we claim is continuous) is injective on \(V\). Indeed, since \(f_x\) is continuous, there exists some left-inverse \(g\) of \(f_x\). Since the map
  \[(h \mapsto gh) : \Hom(\R^k, \R^n) \to \R\]
  is continuous, there exists a neighborhood \(V_0\) of \(f_x\) such that for every \(h \in V_0\), \(h\) is injective. Then we can take \(V = \mathbf{s}^{-1}(V_0)\).
\end{proof}

\section{Projective modules}
Throughout this section, we fix a commutative ring \(R\).

\begin{definition} \label{def-left-exact}
  Let \(R\) and \(S\) be rings. We say that a functor \(F : R\Mod \to S\Mod\) is \textbf{left-exact} if it preserves \(0\) and direct sums, and for every short exact sequence \(0 \to A \overset{f_1}{\to} B \overset{f_2}{\to} C \to 0\) in \(R\Mod\),
  \[0 \to F(A) \overset{F(f_1)}{\to} F(B) \overset{F(f_2)}{\to} F(C)\]
  is a short exact sequence in \(S\Mod\)
\end{definition}

\begin{definition} \label{def-right-exact}
  Let \(R\) and \(S\) be rings. We say that a functor \(F : R\Mod \to S\Mod\) is \textbf{left-exact} if it preserves \(0\) and direct sums, and for every short exact sequence \(0 \to A \overset{f_1}{\to} B \overset{f_2}{\to} C \to 0\) in \(R\Mod\),
  \[F(A) \overset{F(f_1)}{\to} F(B) \overset{F(f_2)}{\to} F(C) \to 0\]
  is a short exact sequence in \(S\Mod\).
\end{definition}

\begin{lemma}\label{lemma-left-exact-finite-limits}
  \(F : R\Mod \to S\Mod\) is left exact if and only if \(F\) preserves finite limits.
\end{lemma}

\begin{lemma}\label{lemma-right-exact-finite-limits}
  \(F : R\Mod \to S\Mod\) is right exact if and only if \(F\) preserves finite colimits.
\end{lemma}

\begin{lemma}\label{lemma-covariant-hom-left-exact}
  Let \(M\) be an \(R\)-module. Then the covariant hom-functor \(\RHom{R}{M}{-}\) is left-exact.
\end{lemma}

\begin{proof}
  We need to show that for every exact sequence \(0 \to A \overset{f_1}{\to} B \overset{f_2}{\to} C \to 0\),
  \[0 \to \RHom{R}{M}{A} \overset{f_1 \circ -}{\to} \RHom{R}{M}{B} \overset{f_2 \circ -}{\to} \RHom{R}{M}{C}\]
  is exact.

  That is, we need to show that
  \begin{enumerate}
  \item \label{lemma-covariant-hom-left-exact-1}\(f_1 \circ -\) is injective.
  \item \label{lemma-covariant-hom-left-exact-2} \(\im(f_1 \circ -) = \ker(f_2 \circ -)\).
  \end{enumerate}

First, we prove  \ref{lemma-covariant-hom-left-exact-1}.
    Let \(g_1, g_2 : M \to A\) such that \(f_1 \circ g_1 = f_1 \circ g_2\). By assumption on \(f_1\), \(f_1\) is injective, so it follows that \(g_1 = g_2\).

Now we prove \ref{lemma-covariant-hom-left-exact-2}.
    Let \(g : M \to A\). Then \(f_2 \circ f_1 \circ g : M \to C\) is zero because \(f_2 \circ f_1\) is zero, which shows that \(\im(f_1 \circ -) \subseteq \ker(f_2 \circ -)\). For the other inclusion, let \(h : M \to B\)  such that \(f_2 \circ h = 0\). It suffices to exhibit a \(g : M \to A\) such that \(f_1 \circ g = h\). We define \(g\) pointwise. Fix \(m \in M\). Since \(f_2 \circ h = 0\), we have that \(\im(h) \subseteq \ker(f_2)\). By assumption on \(f_2\), \(\ker(f_2) = \im(f_1)\), so \(\im(h) \subseteq \im(f_1)\). That is, there exists some \(a \in A\) such that \(h(m) = f_1(a)\). We define \(g(m) = a\). Since \(f_1\) is injective, this function is well-defined and is easily checked to be an \(R\)-module homomorphism.

  \end{proof}

  \begin{definition}[\cite{nlab:projective_module}]\label{def-projective}
  Let \(M\) be an \(R\)-module. \(M\) is projective if it satisfies any of the following properties:

  \begin{enumerate}
  \item\label{projective-lifting} For any \(A, B\), \(f : M \to B\) and \(g : A \twoheadrightarrow B\), there exists a lift \(\widetilde{f}\):
    \[
      \begin{tikzcd}
        & A \arrow[two heads]{d}{g}\\
        M \arrow[dashed]{ur}{\widetilde{f}}\arrow[swap]{r}{f}& B
      \end{tikzcd}
    \]
  \item \label{projective-free-summand} \(M\) is a direct summand of a free module.

    \item \label{projective-split} Every short exact sequence \(0 \to A \to B \to M \to 0\) is split.

    \item \label{projective-hom-exact} The covariant hom-functor \(\Hom_{R\text{-Mod}}(M, -)\) is exact.
  \end{enumerate}
\end{definition}

\begin{lemma}\label{lemma-projective-equivalent}
  The properties \ref{projective-lifting}, \ref{projective-free-summand}, \ref{projective-split}, and \ref{projective-hom-exact} are equivalent.
\end{lemma}
\begin{proof}
  \begin{itemize}
  \item (\ref{projective-lifting} \(\implies\) \ref{projective-hom-exact})
    Let \(0 \to A \overset{f_1}{\to} B \overset{f_2}{\to} C \to 0\) be an exact sequence. We need to show that the following sequence is also exact:
    \[0 \to \RHom{R}{M}{A} \overset{f_1 \circ -}{\to} \RHom{R}{M}{B} \overset{f_2 \circ -}{\to} \RHom{R}{M}{C} \to 0.\]
    Left-exactness is exactly the conclusion of \ref{lemma-covariant-hom-left-exact}.

    To show right-exactness, it suffices to show that \(f_2 \circ -\) is surjective. Fix \(h : M \to C\). By assumption, \(f_2\) is surjective. Then we are in the situation of \ref{projective-lifting}, so there exists a lift \(\wt{h} : M \to B\) of \(h\) along \(f_2\) such that \(f_2 \circ \wt{h} = h\).

    \item (\ref{projective-lifting} \(\implies\) \ref{projective-split}) Suppose that \(0 \to A \to B \overset{f}{\to} M \to 0\) is exact. Then \(f\) is an epimorphism, so there exists a lift \(g\) of the identity \(\operatorname{id} : M \to M\) along the surjection \(f\), such that \(f \circ g = \operatorname{id}\), so the sequence is split.
  \item \TODO(jesse)
  \end{itemize}
\end{proof}

\begin{lemma} \label{lemma-fg-projective-idempotent}
  Let \(M\) be a finitely-generated projective \(R\)-module . Then there exists an \(n\) and an idempotent \(e : \Mat_n(R)\) such that \(M \simeq e(R^n)\).
\end{lemma}

\begin{proof}
  \TODO(jesse).
\end{proof}

\begin{definition} \label{def-category-projective-modules}
  We define \(\fgProj(R)\) to be the full subcategory of \(R\text{-Mod}\) on the finitely-generated projective \(R\)-modules.
\end{definition}

\section{Swan's theorem}

%TODO(jesse): maybe explicit construct and name the inverse functor?
\begin{theorem}[Swan, \cite{swan1962vector}]
  The global sections functions induces an equivalence of categories
  \[\Gamma : \Vect(X) \overset{\sim}{\longrightarrow} \fgProj(X).\]
\end{theorem}

\begin{proof}
  \TODO(jesse)
\end{proof}

\section{Kaplansky's theorem on projective modules over local rings}

\begin{definition}\label{def-local-ring}
  Let \(R\) be a commutative ring. \(R\) is \textbf{local} if it has a unique maximal ideal.
\end{definition}

\begin{definition}\label{def-free-module}
  Let \(R\) be a commutative ring. Let \(M\) be an \(R\)-module. \(M\) is \textbf{free} if \(M\) has a basis; equivalently, if \(M\) is isomorphic to a direct sum of copies of \(R\).
\end{definition}

\begin{theorem}[Kaplansky]\label{thm-kaplansky}
  Let \(R\) be a local ring, and let \(M\) be a projective module over \(R\). Then \(M\) is free.
\end{theorem}

\begin{proof}
  \TODO(jesse)
\end{proof}

\section{Internalizing Kaplansky's theorem to \(\Shv(X)\)}

\bibliography{swan-notes}

\end{document}