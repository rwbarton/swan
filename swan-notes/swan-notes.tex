\documentclass[11pt]{article}
 
\usepackage[margin=1in]{geometry} 
\usepackage{tikz-cd}
\usepackage{graphicx}
\usepackage{hyperref}
\usepackage{amsmath,amsthm,amssymb,mathrsfs,mathabx}
\usepackage{enumitem}
\usepackage{csquotes}
\usepackage{color} 
\usepackage{parskip}
\usepackage{hyperref}
\usepackage{fixltx2e}
\usepackage{tcolorbox}
\usepackage{listings}
\usepackage{jmhmacros}

\graphicspath{ {/home/pv/Pictures/latex/} }

\begin{document}

\title{Swan's theorem}
\author{Jesse Han}
\date{\today}

\maketitle

\begin{abstract}
We give an account of Swan's theorem characterizing vector bundles over a compact Hausdorff space.
\end{abstract}

\section*{Introduction}
% TODO
\section{Vector bundles}

\section{Projective modules}

\section{Kaplansky's theorem on projective modules over local rings}

\section{Swan's theorem}

\section{Swan's theorem as the internalization of Kaplansky's theorem}

\end{document}